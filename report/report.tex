\documentclass[letterpaper,10pt,twocolumn]{article}

\usepackage[backend=bibtex,style=numeric]{biblatex}
\bibliography{report}

\usepackage{fullpage}

\usepackage{amsmath}
\usepackage{amssymb}
\usepackage{amsthm}
\usepackage{enumerate}
\usepackage{listings}
\usepackage{algpseudocode}
\usepackage{changepage}  %%for adjustwidth macro
\usepackage{verbatim}
\usepackage{float}
\usepackage{graphicx}
\usepackage{epstopdf}
\usepackage{url}

% Cribbed a listing macro that won't split across pages
\lstnewenvironment{nobreakslisting}[1][]%
    {\noindent\minipage{\linewidth}\medskip
        \lstset{basicstyle=\ttfamily\footnotesize,frame=single,#1}}
    {\endminipage}

\title{COMP 7150 --- Data Science Project Report}
\author{
    Hicks, Eric\\
    \texttt{elhicks@memphis.edu}
    \and
    Kelly, Craig\\
    \texttt{cnkelly@memphis.edu}
}

\begin{document}

% Force pdflatex to properly use letter as page size (instead
% of defaulting to A4)
\special{papersize=8.5in,11in}
\setlength{\pdfpageheight}{\paperheight}
\setlength{\pdfpagewidth}{\paperwidth}

\maketitle

\abstract{
A raw data source was mining for data, which was cleaned and filtered into a
usable dataset. This dataset was searched for interesting details based upon
the authors' preconceptions. They were found.
}

%%%%%%%%%%%%%%%%%%%%%%%%%%%%%%%%%%%%%%%%%%%%%%%%%%%%%%%%%%%%%%%%%%%%%%%%%%%%
%%%%%%%%%%%%%%%%%%%%%%%%%%%%%%%%%%%%%%%%%%%%%%%%%%%%%%%%%%%%%%%%%%%%%%%%%%%%

\section{Introduction}

In this paper, the path and results of the authors' dissection of the Steam
platform will be detailed. The reason Steam was chosen for this project, was
due to its large acceptance in the PC space. Steam is a company that sells
PC, Mac, and Linux games for download over the Internet. With weekly sales,
daily releases, a managed library, and no computer limit on games, it is the
largest digital distributor of games and the most widely used. Additionally,
Steam stores metrics on over a 100 million users on nearly 10,000 games,
making it a perfect example of a feature rich dataset. However, as Steam has
no saved repository of data publicly available, one was created for this
project, hereby called the Steam dataset. \cite{steam}


%%%%%%%%%%%%%%%%%%%%%%%%%%%%%%%%%%%%%%%%%%%%%%%%%%%%%%%%%%%%%%%%%%%%%%%%%%%%
%%%%%%%%%%%%%%%%%%%%%%%%%%%%%%%%%%%%%%%%%%%%%%%%%%%%%%%%%%%%%%%%%%%%%%%%%%%%

\section{Steam Dataset}

Although, several external sites keep track of various parts of Steam, none of
them gathers the data already available by web crawling Steam. Of the few
sites that record data not saved by Steam such as Rhekua \cite{rhekua}, none
of them release snapshots of their gathered data. This lead the authors to
manually mine data from the main Steam site.

TODO: appendix with field listing? or just a link to the repo?

\subsection{Acquisition}

TODO: multiple full pulls from steam and merge with steamspy

\subsection{Cleaning and Formatting}

TODO: reduced to 78 fields
TODO: explain field removal

\subsection{Caveats}

Some aspects of steam were not able to be captured in the data pulled from
Steam.  Some are not recorded and others can only be accessed by an
administrator account.  The number of owners for each game could not be
acquired, but has been estimated using an outside site \cite{steamspy}.
Recommendations were collected as a total of reviews and as two separate
values for positive and negative reviews.  The sentiment of the reviews that
Steam displays was also not able to be captured.  Steam doesn't store
historical price data, so sale history and price trends can not be analyzed
from the data.  These limitations did not stop the authors from gaining
interesting insight from the data.


%%%%%%%%%%%%%%%%%%%%%%%%%%%%%%%%%%%%%%%%%%%%%%%%%%%%%%%%%%%%%%%%%%%%%%%%%%%%
%%%%%%%%%%%%%%%%%%%%%%%%%%%%%%%%%%%%%%%%%%%%%%%%%%%%%%%%%%%%%%%%%%%%%%%%%%%%

\section{Predictions}

As both authors were familiar with Steam before the start of the project, they
made some predictions of what would be found in the data:

\begin{enumerate}
    \item SteamOS games available on Steam would be a perfect match to Linux
    games available.

    \item The most recommended and the highest rated genre is action.

    \item That Metacritic scores are an inverse bell curve when sorted by
    recommendation, i.e. lower and higher scoring games would have more
    recommendations that games with a middle score.

    \item That free games get more reviews per game both are rated lower than
    paid games.
\end{enumerate}


%%%%%%%%%%%%%%%%%%%%%%%%%%%%%%%%%%%%%%%%%%%%%%%%%%%%%%%%%%%%%%%%%%%%%%%%%%%%
%%%%%%%%%%%%%%%%%%%%%%%%%%%%%%%%%%%%%%%%%%%%%%%%%%%%%%%%%%%%%%%%%%%%%%%%%%%%

\section{Testing}

Both authors ran different models on the data independent of each other to
cover the most ground.  Every feature pair was plotted as a scatter plot to
gain a basic understanding of feature relations alongside reading through the
raw data.  From here other options were tried.


%%%%%%%%%%%%%%%%%%%%%%%%%%%%%%%%%%%%%%%%%%%%%%%%%%%%%%%%%%%%%%%%%%%%%%%%%%%%
%%%%%%%%%%%%%%%%%%%%%%%%%%%%%%%%%%%%%%%%%%%%%%%%%%%%%%%%%%%%%%%%%%%%%%%%%%%%

\section{Results}

The following are the results of the authors' predictions:

\begin{enumerate}
    \item It was verified that all SteamOS games are also the entirety of the
    Linux library on Steam.

    \item The most recommended genre was free to play and not action. The least
    recommended genre was non-game software.

    \item The highest scoring genre was sports instead of action. The lowest
    scoring genre was free to play.

    \item Free games do get more recommendations than paid games, they also are
    rated lower than paid games.
\end{enumerate}

The above leads to the conclusion that free to play games likely have mostly
negative reviews. These findings also lead to the idea that free games receive
mostly negative reviews. Other results pulled from the data not related to the
authors' predictions include:

\begin{enumerate}
    \item Pricing compared to Metacritic scores is mostly uniform.  The gaps for
    certain prices are almost entirely due to Steam's pricing structure.

    \item Pricing compared to user recommendations is also nearly uniform.  There
    is a small increase in recommendations for cheaper games, but it is not
    significant.
\end{enumerate}

%%%%%%%%%%%%%%%%%%%%%%%%%%%%%%%%%%%%%%%%%%%%%%%%%%%%%%%%%%%%%%%%%%%%%%%%%%%%
%%%%%%%%%%%%%%%%%%%%%%%%%%%%%%%%%%%%%%%%%%%%%%%%%%%%%%%%%%%%%%%%%%%%%%%%%%%%

\section{Future Work}

Given a larger timespan to collect data and more sophisticated web crawlers,
more data come be pulled from Steam.  This would allow for most, if not all of
the caveats mentioned before to be removed.  With historical data, price
trends for games of different genres could be predicted along with the time
frames for price drops.  A divided recommendation statistic could be used to
determine if certain genres get more positive reviews than others, as well as
show a correlation (or lack thereof) between Metacritic scores and positive
recommendations.

%%%%%%%%%%%%%%%%%%%%%%%%%%%%%%%%%%%%%%%%%%%%%%%%%%%%%%%%%%%%%%%%%%%%%%%%%%%%
%%%%%%%%%%%%%%%%%%%%%%%%%%%%%%%%%%%%%%%%%%%%%%%%%%%%%%%%%%%%%%%%%%%%%%%%%%%%

\section{Conclusion}

After finding a suitable platform to mine data from, the authors acquired raw
data from Steam.  This data was cleaned, formatted, and processed to find
several interesting results.  While all of the authors' predictions were not
proven in testing, the ones that were wrong proved to be the most surprising.


%%%%%%%%%%%%%%%%%%%%%%%%%%%%%%%%%%%%%%%%%%%%%%%%%%%%%%%%%%%%%%%%%%%%%%%%%%%%%
% Bibliography
%%%%%%%%%%%%%%%%%%%%%%%%%%%%%%%%%%%%%%%%%%%%%%%%%%%%%%%%%%%%%%%%%%%%%%%%%%%%%

\nocite{*}                                 % ensure we show the entire bib
\printbibliography

%%%%%%%%%%%%%%%%%%%%%%%%%%%%%%%%%%%%%%%%%%%%%%%%%%%%%%%%%%%%%%%%%%%%%%%%%%%%
%%%%%%%%%%%%%%%%%%%%%%%%%%%%%%%%%%%%%%%%%%%%%%%%%%%%%%%%%%%%%%%%%%%%%%%%%%%%

\end{document}
